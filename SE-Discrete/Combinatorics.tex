\section{Комбинаторика}

\subsection{Аксиоматика. Размещения, сочетания, перестановки.}

\subsubsection{Правила сложения и умножения}

\begin{theorem}[Правило сложения]
    Пусть имеется возможность выбрать элемент $a$ из $n$"=элементного
    множества $A$,
    а элемент $b$ из $m$"=элементного множества $B$. Тогда выбрать элемент 
    из $A$ или $B$ можно $n + m$ способами.
\end{theorem}

\begin{example}
    Выбрать книгу или видеоигру из 15 книг и 3 видеоигр можно 
    $15 + 3 = 18$ способами.
\end{example}

\begin{theorem}[Правило умножения]
    Пусть имеется возможность выбрать элемент $a$ из $n$"=элементного
    множества $A$ и элемент $b$  из $m$"=элементного множества $B$. 
    Тогда последовательно выбрать $a$ из $A$, а потом выбрать 
    $b$ из $B$ можно $n \times m$ способами.
\end{theorem}

\begin{example}
    Выбрать одну книгу и одну видеоигру из 15 книг и 3 видеоигр можно
    $15 \times 3 = 45$ способами.
\end{example}

\subsubsection{Сочетания, размещения, перестановки}
Пусть имеется набор $A = \{a_1, a_2, \dots, a_n\}$. Попробуем понять, 
каким образом мы можем извлекать $k$ элементов из этого набора.

Во"=первых, мы можем подобно правилу умножения <<последовательно>> извлекать 
элементы из множества. Сначала некоторый первый элемент, затем второй и так далее.
В рамках такого случая также есть два возможных варианта поведения.
\begin{itemize}
    \item мы могли бы <<извлечь>> объект, запомнить его и <<вернуть>> в исходное множество. Очевидно, что в этом случае $k \in N$
    \item мы могли бы <<извлечь>> объект и не <<возвращать>> его в исходное множество. Тем самым,
    производя следующий выбор из оставшихся элементов множества. Очевидно, что в таком случае
    $k \leq n$.
\end{itemize}

Оба этих варианта описывают общий случай, называемый размещениями. Итак.

\begin{definition}
    Размещением из $n$ по $k$ элементов назовем упорядоченный набор
    из $k$ различных элементов из некоторого множества различных $n$ элементов.
\end{definition}

В первом случае при этом мы осуществляем размещение без повторений
и обозначаем это как $A_n^k$. Во втором случае размещение с повторениями
и обозначаем это как $\overline{A_n^k}$.

Стоит отметить о частном случае размещения без повторений "--- перестановке.
В этом случае $k = n$, то есть $A_n^n$.

Теперь рассмотрим другой способ. Мы могли бы брать элементы не последовательно,
а некоторой <<пригоршней>>, то есть не учитывать порядок, в котором эти
элементы были взяты. Введем определение.

\begin{definition}
    Сочетанием из $n$ по $k$ назовем набор из $k$ элементов таких, что
    эти наборы отличаются друг от друга только самими элементами, но не их
    порядком.
\end{definition}

А могут ли сочетания быть с повторениями и без повторений? Да. Что есть 
сочетание без повторений ясно, но что в случае с повторениями? В таком случае
мы предполагаем, что каждый объект множества присутствует в выборке в нескольких
экземплярах.

Сочетание без повторений обозначим как $C_n^k$, сочетание с повторениями 
"--- $\overline{C_n^k}$.

\begin{example}
    В качестве примера возьмем замечательное слово <<ЛЯГУШКА>>. Все буквы
    в нем встречаются один раз. Это пример \textit{последовательного выбора без повторений (размещения без повторений)}.
    
    Мы могли бы очень сильно впечатлиться и сказать <<ЛЯГУУУУУУУУУШКА>>.
    В таком случае мы будем иметь дело с \textit{последовательным выбором с повторениями (размещением с повторениями)}.
    
    Если же мы рассмотрим наше слово в виде $\{\text{Л, Я, Г, У, Ш, К, А}\}$ (в виде набора букв)
    и для нас не будет иметь значения, в каком порядке эти буквы были выбраны, то мы будем
    иметь дело с \textit{сочетаниями без повторений}. 
    Понятно, что в таком случае между
    наборами $\{\text{Л, Я, Г, У, Ш, К, А}\}$ и $\{\text{Г, У, Л, Я, Ш, К, А}\}$ нет никакой разницы.
    
    Аналогично мы могли бы рассмотреть и случай сочетаний с повторениями.
\end{example}

Теперь получим формулы для вычисления числа возможных способов
для каждого случая.

\begin{theorem}[О числе размещений с повторениями]
    $\overline{A_n^k} = n^k$
\end{theorem}

\begin{proof}
    Пусть имеется некоторое множество $A = \{a_1, a_2, \dots, a_n\}$
    и мы хотим извлечь из него $k$ элементов с повторениями. Этот процесс
    представляет из себя последовательный выбор одного элемента из $n$ возможных
    (т.к элемент после выбора <<возвращается на место>>). Этот случай
    в чистом виде представляет из себя правило умножение. Получим:
    \begin{equation*}
        \displaystyle \overline{A_n^k} = \underbrace{n \times n \times n \times \dots}_{\text{k раз}} = n^k
    \end{equation*}
\end{proof}

\begin{theorem}[О числе размещений без повторений]
    $\displaystyle A_n^k = \frac{n!}{(n - k)!}$
\end{theorem}

\begin{proof}
    Пусть имеется некоторое множество $A = \{a_1, a_2, \dots, a_n\}$.
    Выберем из него первый из $k$ элементов. В отличие от размещения с повторениями
    мы не будем <<возвращать>> элемент обратно в исходное множество.
    Тогда, оставшиеся $k - 1$ элемент мы можем выбрать $n - 1$ способами.
    Продолжая эту логику далее для оставшихся $k - 1$ элементов получим, что
    \begin{equation*}
        \displaystyle A_n^k = n \times (n - 1) \times (n - 2) \times \dots \times (n - k + 1) = \frac{n!}{(n - k)!}
    \end{equation*}
\end{proof}

Отметим очевидный факт, что число перестановок $\displaystyle A_n^n = n!$.

\begin{theorem}[О числе сочетаний без повторений]
    $\displaystyle C_n^k = \frac{n!}{k!(n - k)!}$
\end{theorem}

\begin{proof}
    Рассмотрим некоторое множество $A = \{a_1, a_2, \dots, a_n\}$. Зафиксируем в нем конкретный набор $C_n^k = \{a_1, a_2, \dots, a_k\}$.
    Отметим, что $k$"=перестановкой этого сочетания мы получим 
    число размещений этого набора. То есть, $C_n^k \times k! = A_n^k$.
    Тогда
    \begin{equation}
     C_n^k = \frac{n!}{k!(n - k)!}
    \end{equation} 
\end{proof}

\begin{theorem}[О числе сочетаний с повторениями]
    $\displaystyle \overline{C_n^k} = C_{n + k - 1}^k$
\end{theorem}

\begin{proof}
    Воспользуемся известным методом доказательства с помощью 
    <<шариков с перегородками>>. Для некоторого множества $A = \{a_1, a_2, \dots, a_n\}$
    зафиксирует $k$"=сочетание с повторениями. Каждый элемент множества $a$ может 
    встретиться в этом сочетании от $0$ до $k$ раз. 
    Запишем цифру $1$ столько раз, сколько раз $a_1$ встречается
    в сочетании. После всех таких единиц поставим $0$. 
    
    Затем запишем цифру
    $1$ столько раз, сколько раз встречается $a_2$, после чего снова
    поставим $0$. Продолжим эту операцию до $a_n$, при этом не будем ставить $0$
    ни перед $a_1$, ни после $a_n$.
    
    Получим некоторую последовательность: $\underbrace{11\dots1}_{\text{число a_1}}0
    \underbrace{1\dots1}_{\text{число a_2}}0\dots\underbrace{11\dots1}_\text{число a_n}$

    Эта последовательность содержит $n - 1$ нулей и $k$ единиц. Тем самым, 
    мы получили биекцию между множеством всех возможных $k$"=сочетаниями с повторениями и множеством последовательностяей
    с заданными нами условиями. 
    
    Число интересующих нас сочетаний при этом в точности
    равно числу последовательностей длины $n - k + 1$, в которых ровно $k$ единиц.
    Осталось отметить, что это число равно $C_{n - k + 1}^k$
\end{proof}

\begin{theorem}[Принцип Дирихле]
    Пусть имеется $n + 1$ <<кроликов>> и $n$ <<ящиков>>. 
    Если всех <<кроликов>> рассадить по <<ящикам>>, 
    то найдется <<ящик>> с не менее двумя <<кроликами>>.
\end{theorem}

Принцип Дирихле во многом является основополагающим при доказательстве многих
нетривиальных фактов из мира математики. Основным навыком при этом является
умение придумать, что в конкретной задаче является <<кроликами>>, а что
<<ящиками>>.

\subsection{Некоторые из классических комбинаторных задач}

\subsubsection{Комбинаторика разбиений}

\begin{theorem}[О разложении предметов по двум ящикам]
    Имеется $n_1$ предметов одного сорта, $n_2$ "--- другого, \dots , $n_k$ "--- $k$"=го сорта. Сколькими
способами можно разложить их в два ящика?

\textbf{Ответ:} $N = (n_1 + 1)(n_2 + 1) \dots (n_k + 1)$

\end{theorem} 

\begin{proof}
Так как в каждый ящик можно поместить от $0$ до
$n_i$ предметов $i$"=ого сорта, а во второй все оставшиеся, по правилу произведения
получим $(n_1 + 1)(n_2 + 1) \dots (n_k + 1)$.
\end{proof}

\begin{corollary}
    Если все предметы различны ($n_1=n_2=\dots=n_k=1$), то их можно разложить $2^k$
способами. 
\end{corollary}

\begin{corollary}
    Если в каждый ящик нужно положить не менее s_i предметов i"=го сорта, то
получим формулу: $(n_1-2s_1+1) \times (n_2-2s_2+1) \times \dots  (n_k-2s_k+1)$. 
\end{corollary}

\begin{theorem}
    Даны $n$ различных предметов и $k$ ящиков. 
Надо положить в первый ящик $n_1$
предметов, во второй "--- $n_2$, \dots, в $k$"=ый "--- $n_k$, 
где $n_1 +\dots + n_k = n$. Сколькими способами можно
сделать такое распределение, если нас не интересует порядок 
предметов в ящике? 

\textbf{Ответ:} $\displaystyle N =  \frac{n!}{n_1!~n_2!\dots n_k!} = \overline{P}(n_1, \dots, n_k)$.

\begin{proof}
    Выложим все предметы в один ряд. Это можно сделать $n!$ способами. Первые $n_1$
предметов положим в первый ящик, вторые $n_2$ предметов "--- во второй ящик, \dots $k$"=ые $n_k$
предметов "--- в $k$"=ый ящик. 

Так как нас не интересует порядок предметов в ящике, то любая
перестановка первых $n_1$ предметов не меняет результат раздела. Точно так же его не меняет
любая перестановка вторых $n_2$ предметов и так далее. 

По правилу произведения
получаем $n_1! ~n_2! \dots n_k!$ перестановок, не меняющих результата раздела. Таким образом, $n!$
перестановок делятся на группы по $n_1!~ n_2! \dots n_k!$ перестановок в каждой группе, причем
перестановки из одной группы приводят к одинаковому распределению предметов. 
\end{proof}

\end{theorem}


\begin{theorem}[О разложении $\displaystyle n_1 = \frac{n}{k}$ предметов в $k$ ящиков]
    Даны $n$ различных предметов и $k$ одинаковых ящиков. Надо положить в каждый
ящик $n_1$ предметов, где $\displaystyle n_1 = \frac{n}{k}$. 
Сколькими способами можно сделать такое распределение,
если не интересует порядок предметов в ящике, и все ящики одинаковы? 

\textbf{Ответ:} $\displaystyle N = \frac{n!}{k!(n_1!)^k}$

\begin{proof}
    Так как $k$ ящиков можно переставить $k!$ способами, а число $\displaystyle n_1 \dots n_k = n_1 = \frac{n}{k}$, то
    по предыдущей теореме число таких комбинаторных объектов равно $\displaystyle \frac{n!}{k!(n_1!)^k}$.
\end{proof}

\end{theorem}

\begin{theorem}[О разложении $n$ одинаковых предметов в $k$ ящиков]
    Сколькими способами можно распределить $n$ одинаковых предметов в $k$ ящиков?

    \textbf{Ответ:} $N = P(n, k - 1) = C_{n + k - 1}^{k - 1}$

    \begin{proof}
        Ситуация моделирует схему, в которой есть $n$ предметов и $k - 1$ перегородок.
        Всего $n + k - 1$ мест, на которых в $i$"=ой позиции размещается либо шар, либо
        перегородка. Число возможных вариантов расставить $k - 1$ перегородок и есть
        искомое число "--- $P(n, k - 1) = C_{n + k - 1}^{k - 1}$.
    \end{proof}
\end{theorem}

\begin{corollary}
    Если в каждый ящик надо положить не менее $r$ предметов, то получим: $P(n - k \cdot r, k - 1)$ способов. 
\end{corollary}

\begin{corollary}
    Если в каждый ящик надо положить хотя бы один предмет, то $r = 1$ и получим
    $P(n - k), k - 1 = C_{n - 1}^{k - 1}$.
\end{corollary}

\begin{theorem}[О разложении $n$ различных предметов в $k$ ящиков]
    Сколько существует способов разложить $n$ различных предметов в $k$ ящиков, если
нет никаких ограничений? 

    \textbf{Ответ:} $n^k$.

    \begin{proof}
        Каждый из предметов можно положить в любой из $k$ ящиков. 
    \end{proof}
\end{theorem}

\begin{theorem}[О разложении $n$ различных предметов в $k$ непустых ящиков]
    Сколькими способами можно поместить $n$ различных предметов в $k$ ящиков, если
не должно быть пустых ящиков? 

\textbf{Ответ:} $k^n - (C^1_k \cdot (k - 1)^n - C_k^2 (k - 2)^n + \dots + (-1)^k \cdot C_k^{k - 1} \cdot 1^n)$.

\begin{proof}
    Данные $r$ ящиков остаются пустыми, если в $k - r$ ящиков предметы
    кладутся без ограничений. $r$ пустых ящиков можно выбрать $C_n^k$ способами.
    В оставшиеся $k - r$ ящиков предметы можно разложить $(k - r)^n$ способами.

    По формуле включений и исключений число распределений при которых
    хотя бы один ящик останется пустым, равно $(C^1_k \cdot (k - 1)^n - C_k^2 (k - 2)^n + \dots + (-1)^k \cdot C_k^{k - 1} \cdot 1^n)$.

    Тогда число распределений, при которых ни один ящик не останется пустым равно
    $k^n - (C^1_k \cdot (k - 1)^n - C_k^2 (k - 2)^n + \dots + (-1)^k \cdot C_k^{k - 1} \cdot 1^n)$.
\end{proof}
\end{theorem}

\begin{theorem}
    Сколькими способами можно распределить $n$ различных предметов по $k$ различным
ящикам с учетом порядка расположения предметов в ящиках, причем все $n$ предметов
должны быть использованы? 

\textbf{Ответ:} $A_{n + k - 1}^n$ способами.

\begin{proof}
    Добавим к $n$ предметам $k - 1$ одинаковых разделяющих предмета.
    Рассмотрим все перестановки из $n$ различных предметов и $k - 1$ одинаковых.
    Таких перестановок $\displaystyle \frac{(n + k - 1)!}{(k - 1)!}$.
\end{proof}
\end{theorem}

\begin{corollary}
    Если не должно быть пустых ящиков, то выберем $k$ предметов и разложим по
одному в каждый ящик. Это можно сделать $A_n^k$ способами. Оставшиеся $n-k$ предметов
разложим в $k-1$ ящик, причем теперь некоторые ящики могут оказаться пустыми.

Такое распределение можно осуществить $\displaystyle \frac{(n - k + k - 1)!}{(k - 1)!}$ способами. Всего
имеем $\displaystyle A_n^k \cdot \frac{(n - 1)!}{(k - 1)!}$ способов.
\end{corollary}

\begin{theorem}
    Сколько существует способов разложить n различных  предметов в 
    k  различных ящиков с учетом расположения предметов в ящиках, 
    если не все n предметов могут быть использованы и некоторые 
    ящики могут оказаться пустыми?
    
    TODO
\end{theorem}

\begin{theorem}[Формула включений"=исключений]
    Пусть $A$ "--- конечное непустое множество.
    $A_1$, $A_2$, \dots $A_n$ "--- система подмножеств
    этого множества. Тогда справедливо равенство:
    \begin{equation*}
        \begin{split}
            |A / \bigcup\limits_{i = 1}^n A_i| = |A| - \sum\limits_{i = 1}^n |A_i| 
            + \sum\limits_{1 \leq i_1 < i_2 \leq n} |A_{i_1} \cap A_{i_2} |
            - \sum\limits_{1 \leq i_1 < i_{2, 3} \leq n}|A_{i_1} \cap A_{i_2} \cap A_{i_3}|
            + \dots 
            + (-1)^r \sum\limits_{1 \leq i_r \leq n} |A_{i_1} \cap A_{i_2} \cap \dots \cap A_{i_r} |
            \\+ \dots 
            + (-1)^n | A_1 \cap A_2 \cap \dots \cap A_n |
        \end{split} 
    \end{equation*}
\end{theorem}

\begin{example}
    Пусть в аудитории сидит 80 человек. В качестве возможных свойств
    для этих людей выберем знание определенного языка: французского, английского
    и немецкого языков. Понятно, что знание одного языка не исключает возможности
    знания другого, а значит множества студентов, знающих разные языки могут пересекаться.

    Тогда $A, F, D$ "--- множества знающих английский, французский и немецкий языки соответственно.

    Предположим, что

    $|A| = 70, |F| = 20, |D| = 30|$

    $|A \cap F = 15|, |A \cap D| = 25, |F \cap D = 7|$.

    $|A \cap F \cap D| = 1$

    Сколько людей не знают ни одного языка?

    $\displaystyle |A / \bigcup\limits_{i = 1}^n A_i| = 80 - 70 - 20 - 30 + 15 + 25 + 7 - 1 = 6$.




\end{example}

\subsection{Рекуррентные соотношения}
\subsubsection{Биномиальные коэффициенты}
\begin{definition}
    Полиномом от $m$ переменных называется выражение

\begin{equation*}
    (x_1 + x_2 + \dots + x_m)^n = (\sum\limits_{i = 1}^m x_i)^n (n, m \in \mathbb{N})
\end{equation*}
\end{definition}

\begin{theorem}[Полиномиальная формула]
    Справедливо равенство:
    \begin{enumerate}
        \item 
        $\displaystyle (a + b)^n = a^n + C_n^1 a^{n - 1} b + C_n^2 a^{n - 2} b^2 \dots b^n = \sum\limits_{i = 0}^n C_n^i a^{n - i} b^i$ (полиномиальная формула)
    \end{enumerate}

    \begin{proof}
        $\underbrace{(x_1 + x_2 + \dots + x_m) \dots (x_1 + x_2 + \dots + x_m)}_{n ~\text{раз}} = x_1^n + x_1^{n - 1} \cdot x_2 + \dots + x_1^{k_1} \cdot x_2^{k_2} \cdot \dots \cdot x_m^{k_m} = \\ = P(k_1, k_2, \dots k_m) \cdot x_1^{k_1} \cdot x_2^{k_2} \dots x_m^{k_m}$
    \end{proof}
\end{theorem}

При этом величины $C_n^k$, возникающие в этих формулах, называются биномиальными коэффициентами
биномиальной формулы.

\begin{corollary}[О сумме полиномиальных коэффициентов]
    Пусть $N$ "--- сумма полиномиальных коэффициентов. Тогда

    $\displaystyle N = \sum\limits_{k_1+\dots+k_m = n} P(k_1, \dots , k_m) = m^n$
\end{corollary}

\begin{corollary}
    Справедливы равенства:
    \begin{enumerate}
        \item $P(k_1, \dots, k_m) = P(k_1 - 1, k_2 \dots, k_m) + P(k_1, k_2 - 1, \dots k_m) +
        \dots + P(k_1, \dots, k_m - 1)$
        \item $\displaystyle \sum\limits_{i = 1}^m x_i = \sum\limits_{k_1+\dots+k_m = n = i} P(k_1, k_2, \dots, k_m) x_1^{k_1} \dots, x_m^{k_m}(x_1 + x_2 \dots +x_m)$
    \end{enumerate}
    
\end{corollary}

Рассмотрим некоторые из свойств биномиальных коэффициентов:
\begin{theorem}[Свойства биномиальный коэффициентов]
    Справедливы равенства:
    \begin{enumerate}
        \item $\displaystyle C_n^k = 
            \frac{n(n - 1)(n - 2) \dots (n - k + 1)}{k!} = P(k, n - k)$
        \item $\displaystyle C_n^k = C_n^{n - k}$
        \item $\displaystyle C_n^k = C_{n - 1}^{k - 1} + C_{n - 1}^k$
        \item $\displaystyle C_n^0 + C_n^1 + \dots C_n^n = 2^n$
        \item $\displaystyle C_n^0 - C_n^1 + C_n^2 - \dots + (-1)^k C_n^k = 0$
        \item $\displaystyle (C_n^0)^2 + (C_n^1)^2 + \dots (C_n^n)^2 = C_{2n}^n$
    \end{enumerate}
\end{theorem}
Доказательства могут быть приведены как комбинаторным путем (с помощью обращения внимания 
на комбинаторный смысл составляющих выражения), так и приведения частей равенства математически.
\subsubsection{Рекурренты}

\begin{definition}
    \textit{Рекуррентным соотношением $k$"=ого порядка} называют
    выражение, задающее $f(n + k)$ через 
    $f(n + k - 1), f(n + k - 2), \dots, f(n)$.

\end{definition}

\begin{definition}
    \textit{Решением рекуррентного соотношения} является числовая последовательность,
    для которой это соотношение тождественно верно.
\end{definition}

\begin{definition}
    \textit{Начальным условиями} называют первые $k$ членов последовательности,
    являющихся решениями рекуррентного соотношения.
\end{definition}

Отметим, что любое решение имеет бесконечное число решений, однако задание
начальных условий позволяет получить однозначное решение.

\begin{definition}
    \textit{Общим решением} рекуррентного соотношения называется
    числовая последовательность, общий член
    которой выражается формулой, зависящей от $k$ произвольных
    постоянных. С помощью подбора постоянных легко можно
    удовлетворить любое начальное условие.
\end{definition}

\begin{example}
    2, 4, 8, 16, 32, \dots "--- $f(n) = 2^n$
\end{example}

\begin{example}
    1, 5, 13, 29, \dots "--- $f(n) = 2^{n + 1} - 3$.
\end{example}

\begin{definition}
    \textit{Линейным рекуррентным соотношением $k$"=ого порядка с постоянными
    коэффициентами} называется соотношение вида 
    \begin{equation}
        c_0 f(n + k) = c_1f(n + k - 1) + c_2 f(n + k - 2) \dots c_k f(n)
        \label{recurr}
    \end{equation}
    здесь $c_i \in \mathbb{N}$ "--- произвольные постоянные, $c_0 \neq 0, c_n \neq 0$.
    
    Если $c_0 = 1$, то такое соотношение называют линейным.

    Линейное однородное рекуррентное соотношение  1"=ого порядка:
    \begin{equation}
        f(n + 1) = cf(n) \label{recurr1}
    \end{equation}

    Линейное однородное рекуррентное соотношение 2"=ого порядка:
    \begin{equation}
    f(n + 2) = c_1 f(n + 1) + c_2 f(n) \label{recurr2}
    \end{equation}
\end{definition}

Далее будем работать с однородными соотношениями для простоты
изложения, в случае неоднородности можно разделить обе части
на коэффициент при старшей степени.

\begin{definition}
    \textit{Характеристическим уравнением} для однородного рекуррентного
    соотношения (\ref{recurr}) называется уравнение вида:

    \begin{equation*}
        r^k = c_1 r^{k - 1} + c_2 r^{k - 2} + \dots + c_k
    \end{equation*}
\end{definition}
Рассмотрим общее решение соотношения 1"=ого порядка.
\begin{theorem}[Общее решение линейный рекуррентных соотношений 1"=ого порядка
    с постоянными коэффициентами]
    Рассмотрим соотношение вида (\ref{recurr1}).

    Тогда общее решение: $f(n) = c^{n - 1} f(1)$.
\end{theorem}

\paragraph{Свойства решений рекуррентного соотношения k"=ого порядка}

\begin{theorem}
    Если последовательность $\{x_n\}$ является решением
    соотношения 2"=ого порядка (\ref{recurr2}), то последовательность
    $\{ \alpha x_n\} ~ \forall \alpha \neq 0$ также является решением рекуррентного соотношения 2"=ого порядка.

    \begin{proof} 
        Если $ x_{n + 2} =  c_1 x_{n + 1} + c_2 x_n$ тождество, то 
        $\alpha x_{n + 2} = \alpha c_1 x_{n + 1} + \alpha c_2 x_n$ также
        является тождеством (для строгости рассуждений можем
        положить $z_n = \alpha x_n$).
    \end{proof}
    \label{solmult}
\end{theorem}

\begin{theorem}
    Если $\{x_n\}$, $\{y_n\}$ "--- решения, то $\{x_n + y_n\}$
    также являются решением рекуррентного соотношения 2"=ого порядка.

    \begin{proof}
        $x_{n + 2} + y_{n + 2} = c_1(x_{n + 1} + y_{n + 1}) + c_2(x_n + y_n)$.

        Положим $x_n + y_n = z_n$ , тогда 

        $z_{n + 2} = c_1 z_{n + 1} + c_2 z_n$ "--- решение (\ref{recurr2}).
    \end{proof}
    \label{solsummary}
\end{theorem}

\begin{theorem}
    Если $r_1$ "--- корень характеристического уравнения однородного
    линейного рекуррентного соотношения 2"=ого порядка (\ref{recurr2}), то
    $\{ r_1^n\}$ "--- решение рекуррентного соотношения 2"=ого порядка.

    \begin{proof} Действительно:
        
        $r_1^2 = c_1 r_1 + c_2$ и, следовательно, $r_1^{n + 2} = c_1 r_1^{n + 1} + c_2 r_1^{n}$.
    \end{proof}
    \label{xar}
\end{theorem}

\subsubsection{Решение однородных линейных рекурретных
соотношений 2"=ого порядка}

\begin{theorem}[Решение однородных линейных рекуррентных соотношений
    2"=ого порядка в случае \textit{различных} корней]
    Пусть $r_1$, $r_2$ различные корни характеристического
    уравнения $r^2 = c_1 r + c_2$. 
    
    Тогда общее решение имеет вид:
    $f(n) = \alpha r_1^n + \beta r_2^n$.

    \begin{proof}
        В силу свойства (\ref{xar}) $\{r_1^n\}$ и $\{r_2^n\}$ являются решениями.

        А в силу свойств (\ref{solmult}), (\ref{solsummary}) $\{\alpha r_1^n + \beta r_2^n\}$ "--- решение.
    \end{proof}

\end{theorem}
\begin{theorem}[Решение однородных линейных рекуррентных соотношений
    2"=ого порядка в случае \textit{одинаковых} корней]

    Пусть уравнение $r^2 = c_1r + c_2$ имеет один корень кратности
    2, то есть $r_1 = r_2$.

    Тогда общее решение имеет вид:
    $f(n) = \alpha r_1^n + \beta n \cdot r_2^n$.

    \begin{proof}
        Пусть $r_1$ "--- корень, то по теореме Виета:

        $\begin{cases}
            c_2 = - r_1^2 \\
            c_1 = 2r_1 \\
        \end{cases}$

        Тогда подставляя $c_1$ и $c_2$ в общий вид решения получим:
        \begin{equation*}
            f(n + 2) = 2r_1 \cdot f(n + 1) - r_1^2 f(n) 
        \end{equation*}

        Покажем, что $n \cdot r_1^n$ "--- решение. Подставим $\{n \cdot r_1^2\}$, тогда:
        
        $f(n + 2) = 2r_1 \cdot f(n + 1) - r_1^2 f(n) = 
        2r_1(n + 1) r_1^{n + 1} - r_1^2 \cdot n \cdot r_1^n = (2n + 2) r_1^{n + 2} - n \cdot r_1^{n + 2} = 
        (n + 2) r_1^{n + 2}$ 
        
        Где $\{r_1^n\}$ и $\{n \cdot r_1^n\}$ "--- решения.

        Тогда $f(n) = \alpha r_1^n + \beta \cdot n \cdot r_1^n$ "--- сумма решений, а значит
        тоже решение.
    \end{proof}
\end{theorem}

\textbf{Замечание:}

Из вышесказанного следует, что заключение $\alpha r_1^n + \beta r_2^n = (\alpha + \beta) r_1^n$ неверно.

\begin{example}
    $f(n + 2) = 4f(n + 1) - 4f(n)$

    Характеристическое уравнение: $r^2 - 4r + 4 = 0$.

    Корни $r_1 = r_2 = 2$.

    Тогда $f(n) = \alpha \cdot 2^n + \beta \cdot n \cdot 2^n$.

    Пусть существуют начальные условия $f(1) = 0$, $f(2) = 8$, тогда

    $\displaystyle \begin{cases}
        2\alpha + 2\beta = 0 \\
        4\alpha + 8 \beta = 8 \\
    \end{cases} \Longrightarrow 
    \begin{cases}
        \alpha + \beta = 0 \\
        \alpha + 2\beta = 2 \\
    \end{cases} \Longrightarrow
    \begin{cases}
        \alpha = -2 \\ 
        \beta = 2 \\
    \end{cases}$ 

    И тогда $f(n) = -2 \cdot 2^n + n \cdot 2^{n + 1} = (n - 1)2^{n + 1}$.
\end{example}

\begin{theorem}[Общее решение однородных линейных рекуррентных соотношений $k$"=ого порядка]
    Обозначим как $P_d(n)$ произвольный многочлен
    $c_d n^d + c_{d - 1} n^{d - 1} + \dots + c_0$, 
    степени $d$ или меньше (то есть $\forall~ 0 \leq i \leq d ~c_i$ может быть равно нулю).

    Тогда общее решение представимо в виде: $y_n = P_{k_1 - 1}(n) \rho_1^n +
    \dots + P_{k_r - 1}(n) \rho_r^n$, где $\rho_1 \dots \rho_n$ "--- представители
    различных корней характеристического уравнения, $r$ "--- количество
    таких представителей.

    \begin{proof}
        TODO.
    \end{proof}
\end{theorem}

\subsection{Степенные ряды и комбинаторные тождества}

\subsubsection{Формальные степенные ряды}

Идея формальных степенных рядов принадлежит Эйлеру. Была
поставлена задача нахождения бесконечного произведения
$(1 - x)(1 - x^2)(1 - x^3) \dots$.

При этом это произведение рассматривается не как математический объект,
а некоторая формальная запись, не имеющая математической интерпретации.

Иными словами, его совершенно не интересовало, какое число $x$ и из какого
множества оно будет, он пользовался лишь известными алгебраическими правилами.

Последовательно перемножая скобки получаются следующие результаты:

$(1 - x) (1 - x^2) (1 - x^3) \dots = (\textcolor{red}{1 - x - x^2} + x^3) (1 - x^3) (1 - x^4) \dots 
(\textcolor{red}{1 - x - x^2} + x^3 - x^3 + x^4 + x^5 - x^6)(1 - x^4) \dots = \dots (*)$

При этом можно заметить, что та часть, идущая в первой скобке, содержащей
<<промежуточный результат>> в своем начале содержит множители, которые
уже не будут меняться с течением времени. На имеющемся фрагменте
этой увлекательной операции это выражения, выделенные \textbf{красным} цветом. 

Действительно, любые
следующие множители будут содержать $x$ с большей степенью и нигде не возникнет
множителей, содержащих нулевую, первую и вторую степень. 

Таким образом, для любой степени есть лишь конечное число перемножений скобок,
чтобы <<зафиксировать>> конечное значение коэффициента при этой степени.

Продолжая такое умножение мы можем заранее сказать, чему будет равен коэффициент
при некоторой степени:

$(*) = 1 - x - x^2 + x^5 + x^7  - x^{12} - x^{15} \dots $.

Посмотрим, как можно найти коэффициент при $x^n$. $x_n$ получается
перемножением вида ($-x^{a_1})  (-x^{a_2}) \dots (-x^{a_t})$.

При этом если таких множителей четное число, то коэффициент при этом конкретном
слагаемом равен $1$, если же нечетное, то $-1$. Таким образом,
суммарный коэффициент $c = n_\text{чет} - n_\text{нечет}$, где $n_\text{чет}$ "--- 
число разбиений с четным числом слагаемых, $n_\text{нечет}$ с нечетным числом слагаемых.

Подобного рода размышления приводят к многим довольно важным тождествам.

\begin{definition}
    Бесконечную последовательность чисел $(a_0, a_1, \dots a_n)$, 
    которую будем называть формальным степенным рядом.
\end{definition}

С формальными степенными рядами стоит работать именно как с бесконечными последовательностями, 
лишь держа в уме, что это последовательность коэффициентов неформального
степенного ряда $a_0 + a_1 x + a_2 x^2 \dots$. 

\begin{definition}
    Множество всех формальных степенных рядов с определенными операциями сложения ($+$) и умножения ($\cdot$)
    образует алгебру $\mathcal{F}$ формальных степенных рядов.

    Для двух формальных степенных рядов $a = (a_0, a_1, \dots)$ и  $b = (b_0, b_1, \dots)$ определим операцию сложения
    \begin{equation*}
        a + b = (a_0 + b_0, a_1 + b_1, \dots)
    \end{equation*}

    и умножения
    \begin{equation*}
        a + b = (\dots, a_0 b_n + a_1 b_{n - 1} + \dots + a_n b_0)
    \end{equation*}
\end{definition}

Это довольно интуитивные операции, которые имеют место быть, когда мы рассматриваем
неформальные ряды в не формальном смысле, а аналитическом (в контексте математического анализа).

Определив операцию умножения, можно определить операцию деления.

Так, пусть $A = B \cdot C$, тогда при известных $A, B$ можно найти $C$ 
с помощью деления $A$ на $B$.

Согласно уже определенной операции умножения, получим, что

$\begin{cases}
    a_0 = b_0  c_0 \\
    a_1 = b_1 c_0 + b_0 c_1 \\
    a_2 = b_2 c_0 + b_1 c_1 + b_2 c_0 \\
    \dots
\end{cases}$

Отсюда, можно получить значение $c_0 = \frac{a_0}{b_0}$, потребовав, что $b_0 \neq 0$.
Остальные коэффициенты могут быть найдены последовательной подстановкой сверху вниз.

Такая операция аналогично <<делению в столбик>> на $B$.

Из этого вытекает любопытный пример.
\begin{example}
    Рассмотрим ряд $A = (1, 0, 0, \dots)$ и ряд $B = (1, -1, 0 \dots)$.

    Такая операция аналогична делению $1$ на $1 - x$, пусть мы и
рассматриваем формальные ряды. Тогда:

$\displaystyle \frac{1}{1 - x} = 1 + x + x^2 + x^3 + \dots$
\end{example}
Такому представлению соответствует ряд $C = (1, 1, 1, \dots)$.

Таким образом, не включая никакого анализа, получена формула, известная со школы, а именно "--- сумма геометрической
прогрессии. Однако в случае, чтобы это равенство стало корректных в функциональном смысле,
нужно потребовать $|x| < 1$. Кроме того, это ещё и разложение в формулу Тейлора.
КАТАРСИС!!1!1!

\subsection{Производящие функции}

Производящей функцией последовательности $a_0, a_1, \dots a_n$ называется
формальный степенной ряд 

$A(t) = a_0 + a_1 t + a_2 t^2 + \dots = \sum\limits_{n = 0}^{\infty} a_n t^n$.

Как уже было сказано, формальная запись не может иметь значения для конкретного значения
$t$.

Операции сложения, умножения, деления определяются согласно
их определению для формальных степенных рядов.

\begin{definition}
    \textit{Подстановкой} производящей функции $B(t)$ в производящую функцию
    $A(t)$ называется производящая функция $C(t) = A(B(t)) = c_0 + c_1 t + \dots + c_n$,
    которая задает последовательность $\{c_n\}$ такую, что:

    $\begin{cases}
        c_0 = a_0 \\
        c_1 = a_1 b_0 \\
        \dots
    \end{cases}$
\end{definition}

\begin{example}
    Пусть $B(t) = -t$, $A(t) = \frac{1}{1 - t}$, тогда
    
    $\displaystyle C(t) = \frac{1}{1 + t} = 1 - t + t^2 - t^3 + \dots = \sum\limits_{n = 0}^\infty (-1)^n t^n$.
\end{example}

\begin{lemma}[О разложении функции $\displaystyle \frac{1}{(1 - \alpha x)^m}$]
    Справедливо равенство:

    $\displaystyle\frac{1}{(1 - \alpha x)^m} = 1 + C_m^1 \alpha x + C_{m + 1}^2 \alpha^2 x^2 + \dots + C_{m + n - 1}^n \alpha^n x^n + \dots$

    \begin{proof}
        Доказательство индукцией (TODO).
    \end{proof}
\end{lemma}

\begin{theorem}[О производящей функции для последовательности, заданной линейным рекуррентным соотношением]
    Пусть $\{a_n\}$ задана линейным рекуррентным соотношением 

    $a_{n + k} = c_1 a_{n + k - 1} + c_2 a_{n + k - 2} + \dots + c_n a_n$.

    Тогда производящая функция для этой последовательности рациональна, причем
    степень многочлена $P(t)$ не превосходит $k - 1$, а $Q(t)$ не превосходит
    $k$. 

    \begin{proof}
        TODO
    \end{proof}
\end{theorem}

\subsubsection{Решение рекуррентных соотношений с помощью производящих функций}
TODO.







